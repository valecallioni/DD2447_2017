% Chapter Template

\chapter{Introduction} % Main chapter title

\label{Intoduction} % Change X to a consecutive number; for referencing this chapter elsewhere, use \ref{ChapterX}

\lhead{1. \emph{Introduction}} % Change X to a consecutive number; this is for the header on each page - perhaps a shortened title

%----------------------------------------------------------------------------------------
%	SECTION 1
%----------------------------------------------------------------------------------------

The magic word model generates $N$ sequences of length $M$, $s^1,\dots,s^N$, where $s^n = s_1^n,\dots, s_M^n$, where all the sequences are over an alphabet $[K].$ Each sequence has a magic word of length $W$ hidden in it, while the rest of the sequence is called background. Our goal is to find $R=r_1, \dots,r_N$, where $r_n$ is the start position of the magic word in the $n$:th sequence $s^n$.

An interesting application of this model can be found in biosequence analysis. The alphabet that we consider is the genomic alphabet: $K = \{ A, C, G, T\}$. The sequences $s^1,\dots,s^N$ become a set of aligned DNA sequences and the positions $[1\dots M]$ are the DNA columns which represent locations along the genome. Our problem can therefore be  reformulated as finding the unknown magic word that appears at different unknown starting positions in those sequences. 

In order to do that, we will implement a collapsed Gibbs sampler to sample from the posterior $p(r_1, \dots, r_N|D)$ where D is the set of DNA sequences generated by the model.
